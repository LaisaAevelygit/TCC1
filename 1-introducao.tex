\chapter{Introdução}



\section{Apresentação}

A Gastrosquise é uma doença neonatal que afeta a parede abdominal do feto, sendo observada em 1:4.000-6.000 nascidos vivos . Tal enfermidade necessita de intervenção imediata, por conta de complicações posteriores que contribuem negativamente para a evolução médica do quadro do recém-nascido. Métodos cirúrgicos têm sido desenvolvidos com a intenção de corrigir o defeito com maior presteza. 


O EXIT-like é um desses métodos, e visa corrigir o defeito da parede abdominal no momento do nascimento da criança. Juntamente com a técnica cirúrgica foi criado um padrão inovador , é o Svelitliza Reducibility Index, criado pelo Dr.Ravier Svetliza, é um índice para verificar e prever o tamanho da alça sentinela abdominal do neonato, vem sendo utilizado para determinar o momento mais apropriado para o parto, utilizado na resolução total da Gastrosquise no “minuto zero”. 


Imagens de Ultrassom são a fonte principal de detecção e posterior correção da doença , pois outros tipos de exames radiológicos mais invasivos não são indicados durante o período gestacional do feto. O trabalho do Dr.Javier Svetliza em  salientar que a prematuridade pode ser fator determinante para o progresso positivo da Gastrosquise é a base deste trabalho. 


Diversas técnicas computacionais tem sido utilizadas para reconhecer padrões de doenças em imagens, tal evolução traz diversos benefícios na área da saúde. A inteligência artificial (IA) é a área da computação responsável pelo trabalho inteligente das máquinas. Dentro da IA temos uma subárea chamada de Deep Learning, esta utiliza as chamadas Redes Neurais Artificiais (RNA) que são modelos matemáticos computacionais que se baseiam nos neurônios humanos para resolverem problemas complexos.




\newpage

\section{Justificativa}

A Neonatologia é uma área médica bastante delicada e sensível, pois trata de cuidar e tratar doenças em recém-nascidos, este universo minúsculo demanda exatidão e perfeccionismo em todas as ações a serem tomadas.

Sobretudo doenças congênitas como a Gastrosquise podem ter um desfecho mais satisfatório com o auxílio da Computação, visando possibilitar a detecção e dimensionamento descritivos dos aspectos anatômicos que compõem a doença.

Com isso,  propõe-se um modelo de Visão Computacional capaz de detectar padrões Gastrosquise em imagens de ultrassom, para realizar o trabalho sobre as imagens será utilizada a Rede Neural Convolucional, que é amplamente utilizada para  esses fins, no que se refere à parte preditiva, relacionado com a  previsão de melhor momento ao parto, faremos uso de Regressão , essa última etapa será embasada no Svelitliza Reducibility Index.


\section{Objetivo}


Este trabalho tem como objetivo desenvolver um modelo computacional capaz de detectar e acompanhar o desenvolvimento da doença fetal Gastrosquise, bem como realizar predições diagnósticas e estatísticas sobre a progressão da anomalia,  com o intuito de alcançar um equilíbrio mais satisfatório entre os índices determinantes para o progresso da doença, que são: Dilatação das alças intestinais e Prematuridade Neonatal.




