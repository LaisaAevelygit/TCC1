
\chapter{Fundamentação Teórica}



Esse capítulo tem como objetivo explicar os conceitos computacionais e médicos, que são necessários ao trabalho, tais como: Gastrosquise e suas técnicas de resolução, Visão Computacional, Tratamento de Imagens, Redes Neurais Artificiais (RNA), Redes Neurais Convolucionais (RNC).

\section{Gastrosquise e Técnicas de Resolução}

A Gastrosquise é uma doença congênita que vem ocorrendo com maior frequencia ao decorrer dos anos
\cite{Gastroschisis}  , caracteriza-se por afetar a parede abdominal fetal , ocorre devido a um defeito no fechamento dessa estrutura resultando na  migração de orgãos para o exterior da cavidade abdominal
\cite{ledbetter}
, sobretudo o intestino. 

Com  a evolução da medicina  e desenvolvimento das técnicas cirúrgicas a Gastrosquise pode ser tratada de maneira mais eficiente, o que trouxe melhores tratamentos para os neonatos e maior taxa de sobrevida."referencia" 


No entanto, ainda na atualidade, a Gastrosquise é uma doença desafiadora, sempre demanda por um atendimento de alta complexidade e internação em UTI Neonatal por vários dias. Condições como prematuridade, baixo peso e complicações cirúrgicas são as maiores causas de letalidade entre os recém-nascidos \cite{artigo_principal}.

Existem dois tipos de Gastrosquise: a simples e a complexa 'referencia',a primeira caracteriza-se por exteriorização das alças intestinais apenas, a outra apresenta complicações como necroses e atresias das áreas envolvidas. O  nivel da patologia é fator relevante nas opções de resolução cirúrgica 'referecia'  , sendo determinante para o fechamento da parede abdominal, de forma primária ou tardia.

\newpage
\subsection{Técnicas Cirúrgicas de Resolução em Gastrosquise}
É necessário entender as opções cirúrgicas que abrangem a Gastrosquise para que possa ser revelado a forma como a computação pode ser utilizada nesse aspecto. As opções são : fechamento primário e fechamento com silo ou tardio. Sendo o fechamento primário uma tendencia , também é a  razão que despertou o interesse em desevolver este projeto.

\subsection{Fechamento Primário}
\subsection{Fechamento Tardio }
\subsection{Exit - like e Svetliza Reducibility Index}
\cite{svetliza}




\newpage
\section{Visão Computacional}
A visão computacional é a forma como computadores´ e máquinas se tornam capazes de enxergar imagens e abstraí-las para um determinado fim, é um processo cognitivo computacional de adquirir informação a partir do processo de formação e transformação das imagens \cite{ballard} , abrange diversas áreas como Processamento digital de Imagens, Reconhecimento de Padrões e IA.

A maneira como os seres humanos conseguem reconhecer  imagens é um processo complexo ; a percepção de luz , intensidade e diferenciação de cores , localização espacial por figuras multidimendionais nos dão uma idéia de como pode ser trabalhosa a tarefa da visão computacional.


Os sistemas de Visão Computacional são destinados basicamente à identificação e classificação de objetos em imagens , atuam tratando as cenas adquiridas , interpolam dados resultantes da abstração das imagens e criam os padrões que serão utilizados nos processos de visão.

Diversos trabalhos tem sido desenvolvidos no sentido de utilizar a visão computacional como ferramenta para automatizar tarefas médicas \cite{avancos}
, a detecção precoce de doenças sempre foi fundamental em prognósticos favoráveis. Tais métodos de trabalhar sobre imagens médicas tem sido amplamente desenvolvidos   \cite{medical}, com a intenção de colaborar com a implementação da computação na medicina.




\section{Pré-processamento de imagens}
Uma imagem pode ser definida como um conjunto de pixels 
\cite{algorithmis}


\section{Redes Neurais}
\cite{deep}
\subsection{Redes Neurais Convolucionais}






